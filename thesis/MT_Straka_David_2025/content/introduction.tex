The online reservation service Reservio~\cite{reservio} claims that 70\% of people prefer to book online, and providing online bookings leads to a 30\% profit increase for businesses. One of the reviews showcased on their website~\cite{reservio} from the University Hospital Brno cites that these services save the hospital over 10 hours a week of administrative work.

A 2019 United States healthcare report by KPMG~\cite{kpmg_healthcare_2030}, on the other hand, claims that \enquote{most consumers prefer to book appointments by phone,} but it also states that about 40\% are unable to do so on the first try and that 58\% of millennials and 64\% of people belonging to the generation X \enquote{value online booking to the extent that they would switch providers in order to do so.}

According to a 2014 study on the adoption, use, and impact of electronic booking in private medical practices in Canada~\cite{pare_medical_ebooking}, both patients and physicians showed growing interest in such system, \enquote{great majority of patients said that they appreciated the system mainly because of the benefits they derived from it, namely, scheduling flexibility, time savings, and automated reminders that prevented forgotten appointments,} and the study's findings \enquote{suggest that the system's automated reminders help significantly reduce the number of missed appointments.}

Yet, in practice, one can see many businesses still opting not to offer online bookings, instead relying mainly on phone calls or sometimes emails for reservations.

When businesses do offer online bookings, they tend to use a wide variety of different systems, each with its own user interface (UI) and user experience (UX). This can be confusing for the customers and can lead to a suboptimal UX due to, for instance, having to learn to use a new UI and familiarizing oneself with the features available within the system, having to manage many different user accounts' login credentials (oftentimes leading to password reuse and thus also posing a security risk), and keeping track of all the bookings spread across the different platforms. Smaller booking systems also tend to struggle with handling surges in traffic. Because of the lack of standardization, migrating from one system to another can prove to be a challenge for both the business and the customers, risking vendor lock-in.

An alternative to having many different booking systems could be one or a few of them becoming dominant within its market. This does have the advantages of a unified UX and fewer login credentials; however, there are many disadvantages and risks associated with such dominant platforms (often dominant to the point of them becoming monopolies or oligopolies). Such risks can be seen in many other types of online services, such as social media, search engines, and e-commerce platforms. Once these dominant platforms form, they can be very difficult for users to escape, for instance, due to the network effect where~\cite{investopedia_network_effect} \enquote{increased numbers of people improve the value of a good or service,} and which~\cite{investopedia_network_effect} also says may lead to less innovation.

% https://www.grammarly.com/blog/spelling-e-mail-email/
With the rise of these dominant platforms, there has also been a growing number of efforts to create decentralized alternatives. Such decentralized systems often provide many of the advantages of a dominant centralized platform, such as being able to interact with a large network of users from a single client application with a familiar UI/UX and a single user account, but without many of the risks associated with the system being run by a single legal entity. An example of such an emerging decentralized system is the Fediverse. Some great examples of older decentralized online systems that are nowadays ubiquitous are email, and even the World Wide Web itself.

The goal of this thesis is to explore the possibility of creating a decentralized, general-purpose online reservation system by designing an open architecture of such system, including a common HTTP-based client-server protocol, and developing a reference implementation including both a client application and a back end. The created solution should be easy to integrate with state-of-the-art cloud-native architectures and should support a use case for a selected type of reservation business. Possible extensions for the selected use case should be discussed. The reference implementation should be tested and evaluated for the selected use case. Additionally, the thesis will explore select existing online reservation systems.
