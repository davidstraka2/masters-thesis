The goal of this thesis was to design an open and decentralized, general-purpose online reservation system architecture, including a common HTTP-based client-server protocol, and developing a reference implementation including both a client application and a back end. The created solution was to be easy to integrate with state-of-the-art cloud-native architectures, and was to support a use case for a selected type of reservation business. Possible extensions for the selected use case were to be discussed. The reference implementation was to be tested and evaluated for the selected use case.

First, a research/overview of select existing online reservation systems was conducted in chapter~\ref{part:existing_reservation_systems}. Based on the results of the research, the assignment of the thesis, as well as original ideas, a requirements analysis was made in chapter~\ref{part:analysis}, including both functional and non-functional requirements for the system to be designed and implemented. Chapter~\ref{part:design} describes the design of the newly created system, including its architecture, the common protocol, a use case for the system, the back-end services, and the client application. Finally, chapter~\ref{part:implementation} describes the implementation of the system based on the created design, as well as its testing and evaluation, fulfilling the selected use case.

The research on the existing online reservation systems could additionally be useful to those who are in the near future looking for a reservation system. Furthermore, together with the requirements analysis, this information could be useful to those who are looking into developing a reservation system of their own. The approaches in the design and implementation chapters can, in part, be used as a reference for those who are looking into developing other applications with some of the technologies or techniques used here. The designed architecture and protocol can be used by others to create new interoperable implementations. And, lastly, the implemented prototype can be further developed into a production-ready system, or integrated into existing infrastructures, as it is released under a permissive open-source license.

One can dream of a future where, if the ideas proposed in this thesis ever became a commonplace reality, after searching for a hairdresser nearby using a search engine, users would be presented with a list of results including not only the basic information about hair salons within close proximity, but also with real-time information about the hairdressers' availability (based on which the results could also be sorted), and a direct link to book an appointment, which would open a user's booking client application of choice, that would in turn automatically fill out the user's personal data if required by the business offering the appointment, and let the user just confirm the booking. Or a future where users could have their booking service automatically book an appointment with their dentist for regular check-ups twice a year, with the exact time and date chosen based on the availability of both parties and an optional notification about the newly created booking.
